\documentclass[a4paper]{article}
\usepackage{geometry}                % See geometry.pdf to learn the layout options. There are lots.
\geometry{letterpaper}                   % ... or a4paper or a5paper or ... 
%\geometry{landscape}                % Activate for for rotated page geometry
%\usepackage[parfill]{parskip}    % Activate to begin paragraphs with an empty line rather than an indent
\usepackage{graphicx}
\usepackage{amssymb}
\usepackage{epstopdf}
\DeclareGraphicsExtensions{.pdf,.png,.jpg}

\usepackage[cm]{fullpage}

\title{Packing Game - Part A}
\author{
Donnie Smith \\
donnie.smith@gatech.edu
  \and 
Kyle Harrigan \\
kwharrigan@gmail.com
}	
\date{August 28, 2012}                                           % Activate to display a given date or no date

\begin{document}
\maketitle

 \section{Problem Statement}
  
 Short problem statement of Part A
 
 \section{Status}
  - Summary of what has been implemented and debugged and what remains to be done
  
 \subsection{Completed}
 
 \subsection{To Be Done}
 
\section{Computing Minimal Containing Disk}
The minimal containing circle was computed by finding the minimal containing circles for all pairs and some triplets of the circles.
The algorithm functions as follows:
\begin{enumerate}
	\item The current best solution is initialized to an impossibly large radius.
	\item For each pair of circles:
	\begin{enumerate}
		\item The minimum containing circle is computed.  If the radius of that circle is smaller than the current best solution:
		\begin{enumerate}
			\item If it contains all circles, then the current best solution is set to that circle.
			\item Otherwise, the minimal containing circle of all triplets that includes the initial pair are considered.  If that circle contains all circles:
			\begin{enumerate}
				\item If the circle is smaller than the current best solution, then the current best solution is set to that circle.
				\item No further triplets of this pair are considered.
			\end{enumerate}
		\end{enumerate}
	\end{enumerate}
\end{enumerate}

This algorithm is not the most efficient available, but runs in $O(n^4)$ time, which is acceptable for a small number of circles.
The solution to the minimal bounding circle for triplets, which is one of the ten cases of Apollonius' Problem, was adapted from code written by Rasmus Fonseca.
The code was modified to fix a bug in which an invalid solution is found when two of the three circles are vertically aligned.

References:
\begin{itemize}
	\item http://en.wikipedia.org/wiki/Smallest\_circle\_problem
	\item http://mathworld.wolfram.com/ApolloniusProblem.html
	\item http://www.diku.dk/hjemmesider/ansatte/rfonseca/
\end{itemize}

\section{Screen Shot}
 - A screen shot of your app showing disks and the container
 
 \section{References}
  - References (papers, URLs) that you found useful
 

\end{document}  
